\documentclass[11pt,a4paper]{article}

\usepackage[utf8]{inputenc}
\usepackage[francais]{babel}
\usepackage[T1]{fontenc}
\usepackage{amsmath}
\usepackage{amsfonts}
\usepackage{amssymb}
\usepackage{cite}
 
\title{Revue de littérature sur la modélisation cellulaire}

\author{Félix Quinton}

\begin{document}

% Création de la page de titre
\maketitle
\thispagestyle{empty}

\section{Revue de littérature}

La modélisation cellulaire consiste à donner un cadre formel aux interactions d'une population de cellule. L'intérêt d'un tel modèle est qu'il permet de tester des hypothèses sur le comportements des agents étudiés grâce à des simulations, et donc sans mener de véritables expériences, qui peuvent être coûteuses et/ou difficiles à réaliser. 

En biologie, les modèles sont traditionnellement exprimés sous la forme d'équations différentielles, dont la résolution impose des hypothèses : les agents doivent être arbitrairement petit (i.e. un agent doit être représenté par un point dans l'espace) et les agents doivent être uniformément répartis dans le domaine. Or, si ces hypothèses sont valides dans certains domaines (écologie, biochimie), elles ne conviennent pas à l'étude du comportement individuel (déplacement, changement d'état) des cellules ni à l'étude des interactions (agglomération, collision) entre cellules \cite{tamulonis2013cell}.

Pour représenter les agents de manière fidèles, nous leur attribuons des propriétés physiques (masse, élasticité,...) et un programme, qui détermine les processus exécutés par l'agent. Ce programme prend en compte le temps écoulé et les interactions avec d'autre cellules survenues depuis le début de la simulation. Par exemple, les Myxobactéries, qui vivent en essaim, changent de comportement en situation de manque de nourriture, adhérant les unes aux autres pour former une fructification (Kaiser \cite{kaiser2003coupling}).

Le programme d'un agent, qui décrit son comportement, est difficile à traduire formellement. Il s'agit d'un ensemble de relation de cause à effet, dont la traduction en langage mathématique serait laborieuse. En revanche, ce type de problème peut être résolu en décrivant un agent comme un programme informatique, sous la forme d'instructions logiques (SI [...] ALORS, TANT QUE [...], etc ...). Ainsi, les agents sont modélisés comme des micro-machines suivant une procédure. Cette approche, qui sera retenue dans cette étude, est appelée "cell based model" (CBM).

La résolution d'un CBM consiste à laisser évoluer la simulation informatique jusqu'à ce que le système arrive à un état d'équilibre. Cet état d'équilibre peut être défini de différentes manières. Par exemple, dans le modèle de croissance du cancer proposé par Qi et al. \cite{qi1993cellular}, l'équilibre est atteint lorsque la proportion de cellule cancéreuse dans le système se stabilise, tandis que dans le modèle d'agrégation de cellule proposé par Leith et Goel \cite{leith1971simulation}, l'équilibre est atteint lorsque l'énergie totale du système est minimale.
 
Pour mettre en oeuvre le CBM, on peut représenter l'espace comme une lattice, c'est à dire un damier. Dans ce cadre, une cellule est représentée comme un ensemble connexe de case du damier. Ce type de modèle est par exemple utilisé par Longo et al. \cite{longo2004multicellular} pour modéliser le processus de gastrulation, i.e. le stade du développement embryonnaire pendant lequel la frontière externe de l'embryon s'organise en couches superposées de cellules. Cependant, cette approche ne permet pas de représenter avec précision la forme des cellules, et nous contraint à des déplacements discrets. Nous lui préférerons donc un modèle continu.

Dans un modèle continu, une cellule est identifiée par la position $r$ de son centre dans un repère orthonormé. Ses déplacements sont déterminés par les lois physiques du système, généralement les trois lois de Newton. Notons que les cellules évoluent dans un milieu (eau, atmosphère) très visqueux à leur échelle. En conséquence, les cellules en mouvement ont une inertie nulle.

Également, on donne une forme aux cellules. Dans la représentation la plus simple, une cellule est représentée par une sphère et est définie par sa position $r$ et son rayon $R$. Alors, la cellule occupe tout l'espace contenu dans la sphère de centre $r$ et de rayon $R$. A chaque itération de la simulation, la position $r$ est mise à jour pour refléter le déplacement de la cellule. 

Angelani et al. \cite{angelani2009self} propose un modèle pour les déplacements des bactéries \emph{E. coli} utilisant ce type de modélisation. Soit un ensemble $\mathcal{P} = \{1,...,n\}, n \in \mathbf{N}$ de bactéries, chaque bactérie $i \in \mathcal{P}$ se déplacent dans une direction $e_i$, en ligne droite, à l'aide d'un flagelle, pendant une courte période de temps, puis tournent suivant un angle aléatoire, acquérant une nouvelle direction $e_i'$ et reproduit ce schéma périodiquement. La bactérie $i$ est alors soumise à sa force de propulsion $F_i^p$ et aux forces de répulsions causées par les bactéries proches $F_{ij}^c, j \in \mathcal{P}\backslash i$. Grâce à cette modélisation, Angelani a pu montrer que le mouvement erratique des \emph{E. coli} permet d'actionner un micro-moteur.

Dans la modélisation précédente, les cellules ne sont pas autorisées à se déformer. Les forces de répulsions sont telles que l'intégrité spatiale de chaque cellules est conservée, ce qui correspond bien aux \emph{E. coli}. Cependant, les cellules animales se déforment facilement. Elles prennent de préférence une forme sphérique, mais lorsqu'elles sont pressées l'une contre l'autre, elles se déforment et forment une interface. Il est donc important de modéliser des cellules élastiques, pour prendre en compte ce comportement. 

Drasdo \cite{drasdo2005single} propose un modèle de sphères déformables où le rayon $R$ de la sphère représentant une cellule est une variable. Soit $\mathcal{P} = \{1,...,n\}$ un ensemble de cellules et $i,j \in \mathcal{P}, i \ne j$ deux cellules. Si $||r_i - r_j|| < R_i + R_j$, alors les sphères représentant $i$ et $j$ se superposent, et on suppose que la frontière de chacune des cellules s'aplatie contre celle de sa voisine. L'énergie de déformation subie par les deux sphères pressées l'une contre l'autre est donnée par le modèle de Hertz. 

Plus précisément, Drasdo définit l'énergie résultant de l'interaction entre deux cellules $i$ et $j$ comme : $$W_{ij} = W_{i}^K + W_{j}^K + W_{ij}^D + W_{ij}^A  $$

Où $W_{ij}^D$ désigne l'énergie de déformation causée par le contact entre les cellules $i$ et $j$, donnée par le modèle de Hertz, et $W_{ij}^A$ représente l'énergie d'adhésion entre les cellules et entre les cellules et le milieu , qui dépend linéairement de l'aire de contact cellule-cellule et cellule-milieu. Enfin, pour modéliser les variations de volume des cellules et pour prendre en compte leur tendance à retrouver leur volume d'équilibre $V^0_i$ et $V^0_j$, Drasdo introduit dans son potentiel les termes $$W_i^K = \frac{K}{2}(\frac{V_i^k-V_i^0}{V_i^0})^2, \forall i \in \mathcal{P}$$ où $k$ désigne l'itération considérée, $K$ est le coefficient d'élasticité volumique et $V_i^k$ désigne le volume effectif de la cellule $i$ à l'itération $k$. Ainsi, la minimisation du potentiel à chaque itération tendra à rendre aux cellules leur volume d'équilibre.

En plus de la déformation des cellules, une autre propriété des cellules animales que nous souhaitons modéliser est l'adhésion de cellule à cellule, un processus clef dans la formation d'organismes multicellulaires. La membrane des cellules est dotée de molécules adhésives pouvant s'accrocher aux molécules adhésives des cellules voisines, créant une force qui attirent les cellules considérées l'une vers l'autre. Pour modéliser ce phénomène, Galle et al. \cite{galle2005modeling} propose d'exprimer la force d'adhésion entre une cellule $i \in \mathcal{P}$ et une cellule $j\in \mathcal{P}\backslash i$  comme une fonction linéaire de l'aire de contact entre les cellules $i$ et $j$, notée $A_{ij}$. Cette énergie s'écrit alors $W_{ij}^A = \epsilon_k A_{ij}$, où $\epsilon_k$ est l'énergie d'adhésion moyenne par unité de surface. 

Malheureusement, l'utilisation du modèle de Hertz pour le calcul de la force de répulsion engendrée par la collision de deux cellules est limitée, car le modèle de Hertz ne s'applique qu'à de petites déformations et pour seulement deux sphères simultanément. Pour contourner ce problème et représenter des agrégats de cellules, il est courant de s'orienter vers la tessellation de Voronoi. Un exemple de tessellation de Voronoi est le modèle de Schaller et Meyer-Hermann \cite{schaller2005multicellular}. Les auteurs étudient des cellules prenant une forme sphérique au repos, et capables de se déformer si elles sont pressées contre d'autres cellules. Dans ce cadre, une sphère de rayon $R_i$ et un domaine de Voronoi $\mathcal{V}_i^k$ sont calculés pour chaque cellule $i \in \mathcal{P}$, et la représentation de plus petit volume est conservée. 

Cependant, si la tessellation de Voronoi permet de modéliser avec plus de justesse les déformations liées aux forces de répulsion entre cellules, cette méthode est coûteuse en terme de calculs, puisqu'il est nécessaire de recalculer, à chaque itération, la tessellation à partir du centre des cellules. Pour résoudre ce problème, Honda \cite{honda1978description} propose de remplacer le centre $r$ des cellules et leurs dimensions par les sommets $\mathcal{T}$ de la tessellation de Voronoi en tant que variables du modèle. Pour mettre à jour les positions, on spécifie un potentiel d'énergie pour tout le système, puis on applique son gradient aux coordonnées des sommets. 

Un important mécanisme du modèle de Honda est la reconfiguration des frontières des cellules. En se déplaçant, les cellules changent parfois de voisins, et donc des frontières entre cellules doivent être supprimées tandis que d'autres doivent être créées. 

Honda remarque que lorsque la frontière entre deux cellules $i,j \in \mathcal{P}$ raccourci, on déduit que les cellules s'éloignent l'une de l'autre. On en déduit également que les deux cellules $k,l \in \mathcal{P}$ voisines de $i$ et $j$ se rapprochent. Après un certain temps, les cellules $i$ et $j$ ne se toucheront plus, et une nouvelle frontière sera créé entre les cellules $k$ et $l$. Ce phénomène peut être modélisé en remplaçant les arcs très courts de la tessellation, représentant les frontières qui raccourcissent, par des arcs perpendiculaires, représentant de nouvelles frontières entre les cellules qui se rapprochent.

De plus, les cellules doivent également être en capacité de se déplacer d'elles même. En effet, la membrane des cellules subit une pression interne, dirigée vers l'extérieur, et les cellules sont capables de se créer des protubérances, en augmentant l'élasticité d'une partie de leur membrane pour permettre à leur pression interne de repousser celle-ci. Pour modéliser cette possibilité, Weliky et Oster \cite{weliky1990mechanical} propose d'ajouter deux sommets $t'$ et $t''$ de part et d'autres du sommet $t \in \mathcal{T}$, et de relâcher la tension des arc $(i',i)$ et $(i,i'')$. Alors, la pression interne repousse le sommet central $i$, ce qui reproduit le mécanisme naturel décrit précédemment.  

Farhadifar et al. \cite{farhadifar2007influence} étudie l'évolution des cellules épithéliales, qui composent les tissus organiques, comme la peau, en formant des agrégats compacts. L'épithélium étant dense, les cellules qui le composent sont pressées les unes contre les autres, et ont des caractéristiques (forme, aire, nombre de voisins) différentes. De plus, les cellules épithéliales sont également capables de modifier leur forme et leurs voisins pour permettre certain mécanisme métabolique.  

L'approche retenue pour modéliser ce système est un modèle de réseau en 2D décrivant les forces agissant pour déformer et déplacer les cellules. Dans ce cadre, un équilibre du système correspond à un état du système minimisant une fonction d'énergie. Dans le modèle de Farhadifar, le potentiel d'énergie s'écrit : $$E = \sum_\alpha \frac{K_\alpha}{2}(V_\alpha - V_\alpha^0)+\sum_{i,j}\Lambda_{ij}l_{ij} + \sum_\alpha\frac{\Gamma_\alpha}{2}L_\alpha^2 $$
Dans ce potentiel, la première somme désigne l'énergie d'élasticité résultant de la différence entre la surface effective des cellules $V_\alpha , \alpha \in \mathcal{T}$ et leur surface d'équilibre $V^0_\alpha , \alpha \in \mathcal{T}$. Les termes $K_\alpha, \alpha \in \mathcal{T}$ représentent les coefficients d'élasticité des cellules. La deuxième somme décrit les tensions linéaires au niveau des frontières entre deux cellules adjacentes $i$ et $j$. Les termes $l_{ij}$ désignent la longueur de la frontières entre les cellules $i,j \in \mathcal{T}$, et les termes $\Lambda_{ij}$ sont des coefficients représentant la magnitude de ces tensions linéaires. Enfin, la troisième somme représente la contractilité de la membrane des cellules. $L_\alpha$ est le périmètre de la cellule $\alpha$, et $\Gamma_\alpha$ est le coefficient de contractilité associé à la cellule $\alpha$.

Une configuration stable du système correspond à un minimum local de la fonction d'énergie, c'est à dire un point qui vérifie que la force nette appliquée à chaque sommet $i \in \mathcal{T}$ du système est nulle : $$F_i = \frac{\partial E}{\partial R_i} = 0, \forall i \in \mathcal{T}$$ 

Un premier résultat présenté par Farhadifar concerne la géométrie de l'état du système réalisant le minimum global de la fonction d'énergie. En considérant la contractilité normalisée $\overline{\Gamma} = \frac{\Gamma}{KA^0}$ et la tension normalisée $\overline{\Lambda} = \frac{\Lambda}{K(A^0)^{\frac{3}{2}}}$, les auteurs ont été capables d'identifier deux zones.\\ 
Lorsque $\overline{\Gamma}$ est grand, c'est à dire que les forces liées à la contractilité sont grandes par rapport aux forces liées à l'élasticité de la membrane, ou que $\overline{\Lambda}$ est positif, ce qui veut dire que les frontières des cellules ont tendance à rétrécir, alors il existe un unique minimum global qui est tel que toutes les cellules ont une forme hexagonale régulière et une aire est égal à l'aire d'équilibre : $V_i = V^0, \forall i \in \mathcal{T}$. \\
En revanche, si $\overline{\Gamma}$ est petit, et $\overline{\Lambda}$ est négatif, alors il existe une infinité de minima globaux pour lesquels les cellules prennent des formes irrégulières. On sait tout de même que les cellules ont toutes une aire égal à leur aire d'équilibre et un périmètre $L_i = -\frac{\Lambda}{2\Gamma}, \forall i \in \mathcal{T}$. 

La fonction d'énergie admet aussi des minima locaux. Le développement d'un épithélium peut être représenté comme une suite de ces minima locaux, et la transition du système d'un minimum local à un autre peut être causée par des perturbations locales, telles que la division d'une cellule ou l'apoptose (i.e. la mort d'une cellule dans le cadre normal du fonctionnement de l'organisme), ou par la modification graduelle des propriétés des cellules.

Farhadifar propose une procédure pour simuler la division cellulaire. Dans un premier temps, une cellule choisie au hasard pour se diviser voit son aire d'équilibre doublé progressivement. Si une autre cellule venait à être écrasée, c'est à dire que son aire est réduite à 0, au cours de ce processus, alors on considère qu'elle meure et elle est supprimée de la simulation. \\
Après cette première étape, un arc passant par le centre de la cellule par un angle choisit aléatoirement est créé, ce qui divise la cellule considérée en deux, simulant fidèlement le processus de division cellulaire.

Les auteurs vérifient en implémentant leur modèle que le mécanisme de division cellulaire perturbe le système et l'empêche de rester dans une configuration stable. Selon les valeurs des paramètres $\overline{\Gamma}$ et $\overline{\Lambda}$, les résultats des simulations diffèrent sur la forme et sur le nombre de voisins des cellules. Notons que les auteurs parviennent à obtenir des résultats proches de résultats obtenus lors d'expériences réelles avec les valeurs $\overline{\Gamma}=0.04$ et $\overline{\Lambda}=0.12$, qui correspondent à un cas où il existe un minimum global.

Une autre expérience menée par Farhadifar consiste à découper la membrane d'une cellule, à l'aide d'un laser. Ce procédé est simulé en sélectionnant un arc $(i,j)$ au hasard et en fixant sa tension à 0 : $\Lambda_{ij} = 0$. La contractilité des cellules adjacentes est également fixée à 0 : $\Gamma_i=\Gamma_j=0$. Après cette étape, on reprend la minimisation de la fonction d'énergie. Cette expérience permet de retrouver les résultats expérimentaux en utilisant les même valeurs que précédemment pour $\overline{\Lambda}$ et $\overline{\Gamma}$.

\section{Problématique}

Les concepts présentés dans la section précédente ont été implémentés par Gay et al. \cite{gaygithub} dans la librairie Python tyssue, qui a été utilisée pour modéliser le comportement de cellules composant un organoïde cultivé à partir de cryptes issues de la paroi du colon. Les cryptes du colon sont des cavités dont la paroi est couverte de cellules épithéliales et qui sont responsables du renouvellement des cellules de la paroi du colon. Pour mener cette étude, nous nous basons sur des images obtenues par microscopie de fluorescence. Un traitement d'image réaliser par le logiciel spécialisé CellProfiler nous permet de définir un maillage pour modéliser des cellules composant l'organoïde étudié. 
Alors, notre objectif sera de démontrer le potentiel prédictif de notre modélisation en retrouvant des mécanismes fonctionnels, observés lors d'expérience en laboratoire, grâce à notre modèle. 

\section*{Appendix A : Modèle de Hertz}

Le modèle de Hertz, proposé en 1882, est utilisé pour calculer l'énergie générée par de petites déformations de surfaces sphériques. Soient $i$ et $j$ deux cellules sphériques, l'énergie de déformation générée par le contact entre $i$ et $j$ est donnée par $$W_{ij}^D = \frac{2(R_i+R_j-r_{ij})^{5/2}_k}{5D_{ij}}\sqrt{\frac{R_iR_j}{R_i+R_j}} $$
Où $R_i$ et $R_j$ sont les rayons des cellules $i$ et $j$, $r_{ij}$ est la distance entre les centres des cellules $i$ et $j$, et $D_{ij}$ est une fonction des propriétés des matériaux de la cellules : $$D_{ij} = \frac{3}{4}\left(\frac{1-v_i^2}{E_i}+\frac{1-v_j^2}{E_j}\right) $$
Où $E_i$, $E_j$ sont les modules de Young des cellules, c'est à dire une mesure de leur élasticité exprimée en Pascal, et $v_i$, $v_j$ sont les coefficients de Poisson des cellules, qui permettent de caractériser la dilatation des cellules dans la direction orthogonale à celle de la déformation.

\bibliography{bib}{}
\bibliographystyle{plain}

\end{document}
