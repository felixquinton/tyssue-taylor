\documentclass[11pt,a4paper]{article}

\usepackage[utf8]{inputenc}
\usepackage[francais]{babel}
\usepackage[T1]{fontenc}
\usepackage{amsmath}
\usepackage{amsfonts}
\usepackage{amssymb}
\usepackage{cite}

\title{Chapitre 1: Détermination des paramètres initiaux}

\author{Félix Quinton}

\begin{document}

% Création de la page de titre
\maketitle
\thispagestyle{empty}

\section{Détermination des paramètres initiaux}

\subsection{Présentation du modèle symétrique}

Soient $\alpha \in \mathcal{A} = \{1, ..., N_f\}$ des cellules et soit $\mathcal{E}$ l'ensemble des arcs du maillage représentant ces cellules. Considérons la fonction d'énergie :
$$
E = \sum_{\alpha = 1}^n\frac{K_\alpha}{2}(A_\alpha-A_\alpha^0)^2 + \sum_{(i,j) \in \mathcal{E}}\Lambda_{ij}\ell_{ij}
$$
Où $A_\alpha, A_\alpha^0, K_\alpha, \alpha \in \mathcal{A}$ sont respectivement l'aire, l'aire d'équilibre et le coefficient d'élasticité associé à l'aire de la cellule $\alpha$, et où $\Lambda_{ij}, \ell_{ij}, (i,j) \in \mathcal{E}$ sont respectivement la tension linéaire et le périmètre de l'arc $(i,j)$ du maillage.

Dans cette section, notre objectif sera de mettre en évidence une relation entre les paramètres d'optimisation $\Lambda_{ij}, (i,j) \in \mathcal{E}$ et les données du maillage : $R_a$ le rayon de l'anneau apical (i.e. intérieur), $R_b$ le rayon de l'anneau basal (i.e. extérieur), $N_f$ le nombre de cellules, et $A_\alpha^0, \alpha \in \mathcal{A}$ les aires d'équilibre des cellules.

Pour ce faire, nous allons travailler sur le modèle symétrique, c'est à dire que les propriétés de toutes les cellules seront supposées identiques. Ainsi, nous avons à présent : $A_\alpha = A, A_\alpha^0 = A^0, K_\alpha = K, \forall \alpha \in \mathcal{A}$, et nous définissons $\Lambda_a, \Lambda_b, \Lambda_l$ les tensions sur les arcs apicaux, basaux et latéraux.

Nous pouvons réécrire l'énergie comme suit :
$$
E_s = \frac{N_f K}{2}(A-A^0)^2 + N_f\Lambda_b\ell_b + N_f\Lambda_a\ell_a + 2N_f\Lambda_l\ell_l
$$
Cette expression peut être normalisée en divisant par $N_f$ et en fixant $K=1$. On obtient :
$$
\frac{E_s}{N_f} = \epsilon_s = \frac{1}{2}(A - A^0)^2 + \Lambda_b\ell_b + \Lambda_a\ell_a + 2\Lambda_l\ell_l
$$

\subsection{Expression des grandeurs géométriques en fonction des données du maillage}

Sachant le rayon $R_b$ de l'anneau basal et le rayon $R_a$ de l'anneau apical, on peut déduire la longueur des arcs latéraux, basaux et apicaux :

\begin{align*}
    &\ell_l = R_b - R_a \\
    &\ell_a = 2R_a\sin\Big(\frac{\pi}{N_f}\Big) \\
    &\ell_b = 2R_b\sin\Big(\frac{\pi}{N_f}\Big)
\end{align*}

On peut également calculer l'aire des cellules, qui sont représentées par des trapèzes de hauteur $h = \frac{(R_b-R_a)}{2}\cos\Big(\frac{\pi}{N_f}\Big)$:

\begin{align*}
    A = \frac{h(\ell_a + \ell_b)}{2} = (R_b - R_a)\cos\Big(\frac{\pi}{N_f}\Big)(R_b + R_a)\sin\Big(\frac{\pi}{N_f}\Big) \\
    = \frac{(R_b^2 - R_a^2)}{2}\sin\Big(\frac{2\pi}{N_f}\Big)
\end{align*}


En injectant ces expressions dans la fonction d'énergie normalisée, nous obtenons:

\begin{align}\label{normEner}
    \epsilon_s =\frac{1}{2}\Big(A - A^0\Big)^2 + \Lambda_b R_b\sin\Big(\frac{\pi}{N_f}\Big) + \Lambda_a R_a\sin\Big(\frac{\pi}{N_f}\Big) + 2\Lambda_l(R_b - R_a)
\end{align}

\subsection{Expression des tensions linéaires en fonction des données du maillage}

Nous ne pouvons pas utiliser \eqref{normEner} pour exprimer les tensions linéaires $\Lambda_a, \Lambda_b, \Lambda_l$ en fonction de $R_a$ et $R_b$, car $\epsilon_s$ est inconnue. En revanche, on sait qu'il existe des tensions d'équilibre pour tout couple $(R_a, R_b)$. On pause donc le système :

\begin{subequations}{\label{prepartial}}
    \begin{align}
            &\frac{\partial \epsilon_s}{\partial R_a} = -R_a\sin\Big(\frac{2\pi}{Nf}\Big)\Big(A - A^0\Big)+\Lambda_a
            \sin\Big(\frac{\pi}{N_f}\Big)-2\Lambda_l = 0\label{prepartialA}\\
            &\frac{\partial \epsilon_s}{\partial R_b} = R_b\sin\Big(\frac{2\pi}{Nf}\Big)\Big(A - A^0\Big)+\Lambda_b \sin\Big(\frac{\pi}{N_f}\Big)+2\Lambda_l = 0\label{prepartialB}
    \end{align}
\end{subequations}

Pour simplifier ces équations, nous supposons que l'aire d'équilibre des cellules, $A^0$, peut s'exprimer comme un facteur $\alpha \ge 1$ de $A$, c'est à dire $A^0=\alpha A$. Alors le système \eqref{prepartial} devient :

\begin{subequations}{\label{partial}}
    \begin{align}
            &\frac{\partial \epsilon_s}{\partial R_a} = (\alpha-1)A.R_a\sin\Big(\frac{2\pi}{Nf}\Big)+\Lambda_a
            \sin\Big(\frac{\pi}{N_f}\Big)-2\Lambda_l = 0\label{partialA}\\
            &\frac{\partial \epsilon_s}{\partial R_b} = (1-\alpha)A.R_b\sin\Big(\frac{2\pi}{Nf}\Big)+\Lambda_b \sin\Big(\frac{\pi}{N_f}\Big)+2\Lambda_l = 0\label{partialB}
    \end{align}
\end{subequations}

Pour parvenir à exprimer $\Lambda_a$ et $\Lambda_l$ en fonction de $R_a$ et $R_b$, nous fixons $\Lambda_b = 0$. En réarrangeant les équations, on parvient au système :

\begin{subequations}{\label{lambda}}
    \begin{align}
            &\Lambda_a = 2\cos\Big(\frac{\pi}{N_f}\Big)A(\alpha-1)\big(R_b-R_a\big)\label{lambdaa}\\
            &\Lambda_l = \frac{1}{2}A(\alpha-1)\sin\Big(\frac{2\pi}{N_f}\Big)R_b\label{lambdal}\\
            &\Lambda_b = 0\label{lambdab}
    \end{align}
\end{subequations}

Les équations \eqref{lambda} donnent les relations entre $\Lambda_a, \Lambda_l$ et $R_a, R_b$, qui permettent de définir un point initial bien choisi. Fixer $\Lambda_b = 0$ est cohérent puisque l'on s'attend a trouver des tensions linéaires négligeables sur l'anneau basal.

On peut également obtenir grâce à ces équations un ordre de grandeur pour $\alpha$. En effet, puisque les tensions doivent être au plus de l'ordre de $10^2$, on peut déduire de \eqref{lambdaa}, que $\alpha \le \frac{50}{A(R_b-R_a)}+1$, en faisant l'approximation que $ \cos\Big(\frac{\pi}{N_f}\Big) \approx 1$ lorsque $N_f$ est grand. Nous proposons l'expression $\alpha = 1 + \frac{1}{\max_{\alpha \in \mathcal{A}}A_\alpha}$ qui satisfait cette relation, tant que $\frac{50}{R_b-R_a}\ge \frac{\overline{A}}{A_{max}}$.


\end{document}
